\begin{abstract}
\begin{spacing}{1.5}
Development of a semantic web is gaining a lot of traction recently. At the same time, another change is also getting a lot popular on the web - a move from complex SOAP based web services to the simpler RESTful services that work over the existing HTTP infrastructure. RESTful services were born out of an attempt to unify and simplify the creation and consumption of web based services, and to increase its adoption. Adding semantics to web service descriptions is an important step towards a better web that is accessible to both humans and machines alike. With the web taking a definite turn towards RESTful web services from the traditional RPC- like SOAP based services, it is important to devise a semantic description method for such services. However, for these solutions to be really adopted by developers, they should adhere to the RESTful philosophy and provide a low entry barrier. The proposed solution aligns with the REST philosophy and architecture and reuses existing service documentations to double them as machine-readable descriptions. The Microformats-like syntax used by the solution is simple and easy to write and maintain from a developer perspective. Further, the language describes services as resources with attributes instead of using an RPC-like input-operation-output concept used by most of the current solutions. This markup syntax is then extended to add interlinking between RESTful services, enabling automatic discovery and composition of services. The solution should further reduce the entry barrier for developers and thus increase adoption, resulting in a widely accepted standard. The existence of such a standard technique can make the wealth of resources available on the Internet accessible to machines, enabling the creation of intelligent software agents that can get a lot of work done with no to minimal intervention from humans. 
\end{spacing}
\end{abstract}
