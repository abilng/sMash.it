\chapter{RELATED WORK}
\begin{spacing}{1.5}
Many solutions had been proposed for formally describing RESTful web services. These proposals approach the problem  from different directions each providing a novel way of addressing the issue at hand. Most of these solutions were member submissions to the W3C but there is hardly any consensus on one global standard.

\section{Web Service Description Language (WSDL) 2.0}
WSDL 2.0\cite{10} is an extension of the Web Service Description Language (WSDL)that was used to describe traditional SOAP based services. WSDL 2.0 supports describing RESTful services by mapping the HTTP methods into explicit services available at the given URLs. So every resource will translate into 4 or fewer different services: GET, POST, PUT and DELETE. 
      
The advantage of WSDL 2.0 is that it provides a unified syntax for describing both SOAP based and RESTful services. It also has very expressive descriptions where you can define the specific data type, the cardinality and other advanced parameters for each input type. 
   
However, WSDL 2.0 requires RESTful services to be viewed and described from a different architectural platform: that of traditional RPC-like services. This forceful conversion negates many of the advantages of the RESTful philosophy. In addition, the expressiveness of the format comes at the price of losing the simplicity achieved by moving to the RESTful paradigm. These verbose files are not the easiest to be written by hand and also impose a maintenance headache. Hence, WSDL files are typically generated with the help of some tools. Further, WSDL descriptions are external files based on XML syntax that the developer has to create and maintain separately. 

\section{Web Application Description Language (WADL)}
Web Application Description Language (WADL)\cite{11} is another XML based description language proposed by Sun Microsystems. Unlike WSDL, WADL is created specifically to address the description of web applications, which are usually RESTful services. WADL documents describe resources instead of services but maintain the expressive nature of WSDL. 
WADL still has some of the concerns associated with WSDL in that they still requires an external XML file to be created and maintained by the developer. It also results in boilerplate code. 

\section{hRESTS} 
hRESTS\cite{12} is a Microformat that attempts to fortify the already existing service documentations with semantic annotations in an effort to reuse them as formal resource descriptions. The Microformat uses special purpose class names in the HTML code to annotate different aspects of the associated services. It defines the annotations service, operation, address, method, input and output. A parser can examine the DOM tree of an hRESTS fortified web page to extract the information and produce a formal description. 

hRESTS is a format specifically designed for RESTful services and hence avoids a lot of unnecessary complexities associated with other solutions. It also reduces the efforts required from the developer since he no longer needs to  maintain a separate description of the service.   

One downside with hRESTS is that, despite being specifically designed for REST, it still adheres to an RPC-like service semantics. It is still required to explicitly mention the HTTP methods as the operations involved. Moreover, instead of representing the attributes of a resource, it attempts to represent them as input-output as in traditional services. This results in a lot of unnecessary markup.

\section{SA-REST}

Similar to hRESTS, SA-REST\cite{13} is also a semantic annotation based technique. Unlike hRESTS, it uses the RDFa syntax to describe services, operations and associated properties. The biggest difference between them is that SA-REST has some built in support for semantic annotations whereas hRESTS provides nothing more than a label for the inputs and outputs. SA-REST uses the concept of lifting and lowering schema mappings to translate the data structures in the inputs and outputs to the data structure of an ontology, the grounding schema, to facilitate data integration. It shares much the same advantages and disadvantages as the hRESTS format. In addition, since SA- REST is strictly based on RDF concepts and notations, the developer needs to be well aware of the full spectrum of concepts in RDF.

\section{SEREDASj} 
SEREDASj\cite{14}, a novel description method addresses the problem of resource description from a different perspective. While the other methods resort to the RPC-like semantics of input-operation-output, SEREDASj attempts to describe resources in their native format: as resources with attributes that could be created, retrieved, updated and deleted. This helps to reduce the inherent difference between operation oriented and resource oriented systems. The method also emphasizes on a simple approach that provides a low entry  barrier for developers.

One interesting aspect about SEREDASj is that it uses JSON (JavaScript Object Notation)[15] to describe resources. JSON is an easy and popular markup technique used on the web - especially with RESTful service developers. The advantage is that, the target audience is already familiar with the notation used for markup and can reduce friction when it comes to adoption.
         
SEREDASj, however, addresses the documentation and description in the reverse order. You can create the description in JSON first and then generate the documentation from this. This can increase upgrade effort required for existing services and is not very flexible. It is still possible to embed these JSON descriptions into existing documentation but it floats as a data island, separated from the rest of the HTML page. This, again, causes some duplication of data between the documentation and the description and causes a maintenance hurdle. 
\end{spacing}